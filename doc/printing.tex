%\documentstyle[a4wide]{report}
\documentstyle[12pt,a4wide]{report}
\pagestyle{empty}
\newcommand{\amplot}{\mbox{\bf AMPlot}}
\newcommand{\RA}[1]{\mbox{{\em Right-Amiga}--{\bf #1}}}
\begin{document}

{\LARGE\bf Printing Your Graph}
\vspace*{4em}

\amplot\ is capable of producing a PostScript or Encapsulated PostScript 
(EPSF) file (see Sections~3.1.5 and~4.2). It will not produce output directly to a
PostScript printer. If you select EPSF output, you must import 
the file into a suitable desk-top publishing (DTP) program which will allow you to 
rotate and scale the graph for inclusion in a larger document.

If you have selected a standard PostScipt file as the output format, this must be 
sent to the printer using a command from the CLI. Assuming you have a PostScript 
printer attached to the parallel (Centronics) port of your Amiga, you should use 
the following command:
\begin{verbatim}
            COPY <file.ps> TO PAR:
\end{verbatim}
where \verb1<file.ps>1 is the name of the file you have saved from \amplot.
Similarly, if the printer is attaced to the serial (RS232) port, you should use the 
command:
\begin{verbatim}
            COPY <file.ps> TO SER:
\end{verbatim}
Note, that you should {\em not\/} copy the file to PRT:\ (the Amiga printer device) 
since this will insert control codes not required by your PostScript printer. In 
addition, it is important that your current default directory is the one in which 
the file is saved, or that you specify the full path to the file.

For example, suppose that you save a PostScript file called \verb1mygraph.ps1 in a 
directory (drawer) named 
\verb1Work:PSFiles1. Assuming your printer is attached to the parallel port, you 
should type the following command from a CLI:
\begin{verbatim}
            COPY Work:PSFiles/mygraph.ps TO PAR:
\end{verbatim}
Alternatively you could make \verb1Work:PSFiles1 your current directory with the 
command:
\begin{verbatim}
            CD Work:PSFiles
\end{verbatim}
when it would simply be necessary to issue the command:
\begin{verbatim}
            COPY mygraph.ps TO PAR:
\end{verbatim}
substituting the appropriate name for each file you wish to print.

Since the Amiga is multi-tasking, you can keep a CLI window open for this purpose 
while you are running \amplot. You do not need to exit from \amplot\ to print the 
file, just switch windows and type the appropriate command.

Alternatively, you might decide to write a script to copy a specific file (for 
example \verb1temp.ps1) from the RAM:\ disk to your printer. Using ICONX (see your 
Amiga manual) you could then attach an icon to this script file, such that double 
clicking on the icon would print the file. Then, it is simply necessary for you to 
save the PostScript from \amplot\ under your chosen name on the RAM:\ disk and 
double click on the icon to print the file.


\end{document}
