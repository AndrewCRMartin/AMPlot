\documentstyle[a4wide]{article}
\newcommand{\amplot}{\mbox{\bf AMPlot}}
\begin{document}
{\large\bf This document describes the derivation of the equations used in the
{\tt ClipWedge()} routine used in \amplot.}
\vspace{2in}

\begin{tabular}{ll}
Let: & $G_1$ be $\frac{\mbox{sin}\alpha_1}{\mbox{cos}\alpha_1}$ \\
     & $G_2$ be $\frac{\mbox{sin}\alpha_2}{\mbox{cos}\alpha_2}$ \\
\end{tabular}

The lines are described by the following equations:
\begin{eqnarray}
y   & = & G_1 x      \\
y   & = & G_2 x      \\
y   & = & mx + c     \\
y^2 & = & r^2 - x^2
\end{eqnarray}

The intersect of line (3) with radius (1) is obtained by solving:
\begin{eqnarray*}
y                 & = & G_1 x    \\
y                 & = & mx + c   \\
\Rightarrow G_1 x & = & mx + c   \\
x(G_1 -m)         & = & c        
\end{eqnarray*}

\begin{eqnarray}
x & = & \frac{c}{(G_1 - m)}      \\
y & = & \frac{G_1 c}{(G_1 - m)}
\end{eqnarray}

Similarly, for line (3) with radius (2):
\begin{eqnarray}
x & = & \frac{c}{(G_2 - m)}      \\
y & = & \frac{G_2 c}{(G_2 - m)}
\end{eqnarray}

\newpage
The intersect of line (3) with circle (4) is obtained by solving:
\begin{eqnarray*}
y                                   & = & mx + c      \\
y^2                                 & = & r^2 - x^2   \\
\Rightarrow (mx + c)^2              & = & r^2 - x^2   \\
m^2x^2 + c^2 + 2mcx                 & = & r^2-x^2     \\
(m^2 + 1)x^2 + (2mc)x + (c^2 - r^2) & = & 0           \\
\end{eqnarray*}

\begin{equation}
\label{eq:xcirc}
x = \frac{-2mc \pm\sqrt{4m^2c^2 - 4(m^2+1)(c^2-r^2)}}{2(m^2+1)}
\end{equation}
$y$ is then calculated from Eqns. (3) and the value of $x$ from (\ref{eq:xcirc}).
\vspace{2em}

Before solving for the radii, the slopes should be compared with the slope of the 
line; if identical, the lines will intersect at $\infty$.

These equations result in a possible 4 intersection points. The following checks are 
made to select the two sets of coordinates for the end points of the clipped line:
\begin{enumerate}
\item For each radius intersect, reject if distance from centre of circle is greater 
than the radius ($r$) of the circle.
\item For each radius intersect still valid, if it is not at the centre of the
circle, reject if the angle of the point is outside the range $\alpha_1$--$\alpha_2$.
\item For each circumference intersect, reject if the angle of the point is 
outside the range $\alpha_1$--$\alpha_2$.
\end{enumerate}
We should now have 0, 2, or 3 valid intersects. 
\begin{itemize}
\item If 0, the line does not intersect the wedge. 
\item If 3, we have one of two special cases: Either the line passes through 
the centre of the circle (and thus intersects both radii as well as the 
circumference) {\em or\/} the line passes through the radius/circle intersect and 
the radius. If the former, one set of coordinates is set to the origin while the 
other is set to the valid circle intersect; if the latter, the circle intersect is 
discarded and the two line intersect points are used.
\item If 2, we have the standard case: either 2 line intersects, or a line and a 
circunference intersect.
\end{itemize}
Finally, a check is made that the line does not end up being just a point.

\end{document}



