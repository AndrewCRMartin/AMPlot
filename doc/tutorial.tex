%%%%%%%%%%%%%%%%%%%%%%%%%%%%%%% TUTORIAL %%%%%%%%%%%%%%%%%%%%%%%%%%%%%%%%%%

\chapter{Tutorial}
This tutorial will let you experiment with the main features of \amplot\ using 
datafiles supplied with the program.

\section{Getting Started}
For the purposes of this tutorial, you should start \amplot\ from the Workbench. 
If you have not yet installed the program, please follow the installation 
instructions carefully before proceeding.

Open the \amplot\ drawer which has been created by the installation procedure by 
double clicking, then double click on the \amplot\ icon to start the program.

Remember that you may also start \amplot\ from the CLI/Shell should you so wish, but 
that using the Workbench offers the advantage of being able to set various defaults 
through the icon's tooltypes. For instructions on using icon tooltypes, see 
Chapter~\ref{ch:tooltypes}. Before proceeding with this tutorial, you may wish to 
follow the instructions in Chapter~\ref{ch:tooltypes} to set the default directory 
to the {\tt Samples} sub-directory (specify the complete path to the {\tt Samples} 
directory).

When the program starts, it opens an interlaced high resolution screen\index{screen 
display} with two windows. 
The lower small window is used for informational and error messages.
The upper large window is the main graph-plotting window.
If you are using an NTSC Amiga, the message window may be hidden behind the main 
window. However, when messages are displayed, this window will be sent to the front. 
You may send the message window back behind the main window in the usual way, using
the gadget at the top right of the window.






\section{Loading a Datafile}
To load a datafile\index{Loading a datafile},
select the {\bf Open} item\index{Open} from the {\bf Project} menu. The
keyboard shortcut \RA{O} may be used instead. A 
file requester will appear on the screen from which you should select the file 
{\tt demo1.dat} from the {\tt Samples} subdirectory of the \amplot\ directory.
Make sure you load a valid data file and not (for example) a `.info' file.

The exact form of the file requester will depend on the version of the operating 
system you are using. 
\begin{itemize}
\item Under AmigaDOS~V1.3, the Heath file requester will be used. A 
file is selected by clicking once on the filename, then once on the {\bf OK} gadget.
If you cannot see the name of the file you wish to load, drag the slider to the 
right of the list of files until your file becomes visible. To change directories, 
click in the `drawer' gadget, hit \RA{X} to clear the gadget and type in the name of 
the drive and directory from which you wish to load a file. Alternatively, you 
may simply specify a drive or logical name (i.e. a name ending with a `:') and 
single click on a file prefixed with {\tt (dir)} to select a sub-directory.
\item Under AmigaDOS~V2.0, or above, the system-supplied file requester will be used.
You should look in your Amiga manual for an exact explanation of using the system 
file requester. Essentially it is very similar to the Heath requester used under 
AmigaDOS~V1.3, but lets you select a filename by double-clicking on the name. 
Directories in the filelist are highlighted in a different pen colour and have the 
word {\tt Drawer}.
\end{itemize}

When you exit from the file requester, you will immediately see a graph displayed
on the screen.

Note that if you only have one disk drive, a system requester will appear asking
you to replace the Workbench disk since \amplot\ needs to load fonts from the 
{\tt FONTS:}\ directory. If the program posts such a requester, it may be necessary 
to send the Workbench screen to the back again after the requester has been 
satisfied.

When  a particular font is used and read from disk, it is 
cached internally by the program and does not need to be loaded from disk again. 
If you have limited RAM and are using many fonts, you may find yourself running 
short of memory after a long session of using \amplot. If this occurs, you should 
shut down the program and start again. This is fairly unlikely to occur and the
advantages in speed terms of font caching out-weigh this slight problem.







\section{Adding Labels and Titles}
Titles\index{Labels and Titles} may be added to the axes and labels may be placed 
at arbitrary points on the graph. In addition, a title\index{Title}
may be added to the graph. By default, this will appear towards 
the top of the graph in the centre, but it may be positioned anywhere on the 
graph.

To\index{Axis Titles} add titles to the axes, select the {\bf Axis Titles}
item from the {\bf Text} menu. 
A requester will appear on the screen where titles may be entered for the X and Y 
axes. Click in the appropriate string gadgets and type in the required label. The 
default font\index{font} will be Times-Roman at 14pt size.
You may change this by typing the name of the required font and size. 
For further details of using fonts, see Chapter~\ref{ch:fonts}.
Clearly, you may only specify the names of fonts\index{font}
which you have available on your output device.
For Amiga screen display, Times will be displayed if the requested font is not 
available. Under AmigaDOS~V1.3, only a limited set of font sizes is available---if 
an unsupported size is requested, the nearest available size will be shown on the 
screen, but the final output will have the actual size requested. AmigaDOS~V2.0 
supports font scaling and Compugraphic screen fonts may be selected for better 
quality.

Enter data into the requester as follows:

\begin{tabular}{ll}
{\bf XTitle}  & Time          \\
{\bf YTitle}  & Volume        \\
{\bf X Font}  & Times-Roman   \\
{\bf X Size}  & 14            \\
{\bf Y Font}  & Times-Roman   \\
{\bf Y Size}  & 14            \\
\end{tabular}

\noindent Exit from the requester by clicking on the {\bf OK} gadget. The graph 
will be 
re-plotted on the screen with the axis labels as requested\footnote{Note that the
Y-axis is labelled with letters going down the screen---the output on paper will be 
written sideways up the paper.}.

To add a title\index{Title} to the graph, 
select the {\bf Title} item from the {\bf Text} menu.
A requester appears into which you may enter the required title, font name and 
size, as before. In addition, X and Y coordinates are shown. These refer to the 
bottom middle of your title. i.e.\ Your title will be written with the bottom of 
the text on the line specified by the Y coordinate and will be centred on the X 
coordinate. By default, this position is centred across the graph and is 90\% of 
its height. You may switch the title off by clearing the title text gadget. You 
may type in any coordinates you wish and display the title by clicking on the {\bf 
OK}
gadget. An easier way to specify coordinates is simply to click on the graph at the
point where you wish the title to appear. This may be done with all requesters which 
require coordinates to be specified.
Clicking on the {\bf Centre} gadget will reset the title to the default position.

Enter data into the requester as follows:

\begin{tabular}{ll}
{\bf Title}       & Expansion          \\
{\bf Font Name}   & Helvetica-Oblique  \\
{\bf Size}        & 24                 \\
\end{tabular}

\noindent Exit from the requester by clicking on the {\bf OK} gadget.
The graph will be re-plotted on the screen with the new title.

You\index{Extra Labels} may also add arbitrary text labels to the graph. 
This is primarily intended for labelling lines on multi-line graphs, but may be 
used for any purpose. Select the {\bf Extra Labels} item from the {\bf Text} menu. 

A requester will appear in which you may specify a label and coordinates at which 
to place the label. These refer to the bottom left position of your label 
(as opposed to the bottom centre in the case of the {\bf Text/Title} requester). 
Again the font may be specified as before. In addition there is a set of gadgets 
labelled {\bf Next}, {\bf Prev} and {\bf Kill}. These allow you to step through 
any number of labels and, in the latter case, remove a label.

Enter data into the requester as follows:

\begin{tabular}{ll}
{\bf Label} & Change in Volume   \\
{\bf X}     & 2.0                \\
{\bf Y}     & 8.0                \\
{\bf Font}  & Times              \\
{\bf Size}  & 10                 \\
\end{tabular}

\noindent Now click on the {\bf Next} gadget and type in the following:

\begin{tabular}{ll}
{\bf Label} & over time.   \\
{\bf X}     & 2.0          \\
{\bf Y}     & 5.0          \\
{\bf Font}  & Times        \\
{\bf Size}  & 10           \\
\end{tabular}

\noindent You may also simply hit the return key in the Y-position gadget to step to the next
label, but the screen will only be refreshed with the new label appearing if you
click on the {\bf Next} or {\bf Prev} gadgets.

Exit from the requester by clicking on the {\bf OK} gadget. The {\bf Kill 
All} gadget will delete all extra labels and exit the requester.







\section{Changing The Axes}
You may box\index{Boxed Axes} the graph axes, by selecting the {\bf Boxed}
item from the {\bf Axes} menu (abbreviation \RA{B}). You may also choose 
to have a grid\index{Grid Axes} drawn across the graph, by selecting the {\bf Grid} 
item from the {\bf Axes} menu (abbreviation \RA{G}). If you switch on the grid, the 
boxed option will be switched on automatically; however, switching off the grid will 
not automatically switch off the boxed option.

Try these options now. The menu items will have a tick by them when they are 
activated. Selecting the option again will remove the tick mark and the grid, or box, 
will disappear. Finish by switching on the Grid.

You may choose to display log axes \index{Log Axes} by selecting the {\bf LogX} 
and/or {\bf LogY} items from the {\bf Axes} menu. \amplot\ will ensure that your 
graph data is valid for taking logs (i.e.\ all values are greater than zero). If the 
data are invalid, selecting log axes will simply have no effect.
Experiment with these options now, 
ending up with {\bf LogX} switched on and {\bf LogY} switched off. Note once again 
that the labels and title move with the data. Reset the title's position as before.


\amplot\ will attempt to scale\index{Bounds} the axes automatically. 
Sometimes, however, it is 
not possible to get axes which look as aesthetically pleasing as one might hope.
In addition, the automatic bounds calculations do not account for fitted lines or 
error bars. 
Thus, one may override the automatic bounds setting. Similarly, one may wish to 
alter the default\index{Ticks} positioning of tick-marks along the axes.
These settings are changed by the {\bf Set Axes} item from the {\bf Axes} menu.

Select the {\bf Axes/Set Axes} menu item or use the keyboard shortcut
\RA{A}\index{Bounds}. A requester will appear showing the current values for XMin, 
XMax, YMin and YMax. You may alter these as desired, clicking on the {\bf OK} gadget 
when you have finished. You may reset the default automatic boundary calculations 
by clicking on the {\bf Auto} gadget.

Enter data into the requester as follows:

\begin{tabular}{lr}
{\bf XMin} &  0.0 \\
{\bf XMax} & 10.0 \\
{\bf YMin} &  0.0 \\
{\bf YMax} & 60.0 \\
\end{tabular}

\noindent You may also alter the spacing of the tick marks\index{Ticks} for 
non-log axes. 
The text gadgets for tick and sub-tick spacing of log axes will be disabled.
Enter the following data into the requester:

\begin{tabular}{lr}
{\bf Y}     & 10.0   \\
{\bf YSub}  &  5.0   \\
\end{tabular}

\noindent To switch on subticks (short tick marks without labels), click the 
{\bf Sub} checkbox so the checkmark appears

Finally, click on the {\bf OK} gadget; the graph will be re-plotted, with the
Y-axis running up to the new maximum value of 60, altered tick spacing and subticks.
Note how the labels and title move. This is because they are positioned with respect 
to the original data and not the position on the page. To restore the title to its 
default position on the page, select the {\bf Title} item from the {\bf Text} menu 
and click on the {\bf Centre} gadget. Note that changing the axes from linear to log 
or {\em vice versa\/} will reset the boundary and tick calculations.







\section{Graph Style}
Selecting the {\bf Scatter} item\index{Scatter Plot} from the
{\bf Style} menu will remove the 
line-graph and replace it with a scatter plot---try this now. The data points
will be marked with crosses. The marks used for the data points may be changed
using the {\bf Line Setup} item\index{Line Setup} from the {\bf Style}
menu (Abbreviation: \RA{L}).

\amplot\ supports 6 `pens'\index{pens} (these will be explained in more detail in 
Section~\ref{sect:pens}).
A requester will appear with the 6 pen numbers sets of mutual exclude radio buttons
corresponding to the 12 available data point marks and a space corresponding to no 
mark. All the sliders will be positioned under the cross at this stage. Your 
current graph is drawn in Pen~1. Click on the radio button for Pen~1
under the open circle and select the {\bf OK} gadget. The graph will be 
replotted with circles at the data points.

Once again select the {\bf Style/Line Setup}\index{Line Setup} menu item.
Note that down the right hand side of the requester there are gadgets labelled 
{\bf Link}\index{Linking datapoint marks} corresponding to each of the six pens. 
Clicking on any one of these will join the datapoints for the 
corresponding pen---if you select all these gadgets, the Scatter style will be 
the same as the XY style. This feature allows you to read a datafile containing, 
for example, experimental data and a curve fitted to these data. The experimental 
data may then be displayed using unjoined datapoint marks (e.g.\ circles) while 
the fitted curve is displayed using the {\bf Link} option with no datapoint marks.
Exit the requester once again.

If your data contains information about errors\index{Errors},
you may switch error bars on and off using the {\bf Errors} item from the 
{\bf Style} menu (Abbreviation \RA{E}).
Try switching the error bars on and off now. This dataset contains three columns
of data, so the same value is used for both positive and negative parts of the
error bars. Four column datasets may be read specifying separate positive and
negative errors. If any error value (positive or negative) is zero, no bar
will be drawn for this half of the error.

Bar chart and Pie chart styles are also available. These are described in 
Sections~\ref{sect:bar} and~\ref{sect:pie} respectively.






\section{Fitting and Smoothing}
The {\bf Fit} menu gives you 4 types of fitting options. The first of these is 
linear polynomial regression\index{Regression}. Select the {\bf Regression} item.
A requester 
will appear in which you can specify the degree of the polynomial you wish to 
fit. This may be any value between 2 and 20, though must not exceed the number of 
data points in your graph. If this is the case, a message requester will be posted
every time the graph replots to warn you that the degree is too high.
If you select a degree of 2, you will obtain straight line linear regression fitting, 
where the program fits the data to the equation:
$$y=a+bx$$
Higher values add squared, cubed terms, etc.:
$$y=a+bx+cx^2+dx^3+ex^4\ldots$$
Try polynomial degrees between 2 and 10 selecting the {\bf OK} gadget after entering 
the value.

The coefficients for the fitting will be displayed in the message window in the 
order shown  in the above equation.

With this data set, values of 6, or more, will fit the data almost perfectly when 
{\bf LogX} is on and {\bf LogY} is off.
However, the fitted line will drop below the bottom axis of the graph, so it is 
necessary to reset the bounds. Select a polynomial degree of 7, exit the requester 
and reset YMin to $-30.0$. Once again you will need to reset the position of the 
title.

Select the {\bf Off} gadget in the {\bf Fit/Regression} requester to remove the 
fitted line. Select the {\bf Robust} item\index{Robust Fitting}
from the {\bf Fit} menu. No requester appears, 
but a straight line will be displayed similar to that obtained from 
{\bf Fit/Regression} using a degree of 2.
If, however, you switch this option on as well, you will see the lines are 
slightly different. Robust fitting weights against points which lie outside the 
normal distribution. Section~\ref{sect:rob} will show you this in more detail.

Switch off both the {\bf Regression} and the {\bf Robust} fitting options and 
select the {\bf Fourier} option\index{Fourier Transform Smoothing}
from the {\bf Fit} menu. Fourier smoothing 
performs a Fast Fourier Transform (FFT) on the data, filters out high frequency 
changes in the data (this is controlled by the smoothing factor which you 
specify in the requester) and performs the inverse FFT. Smoothing factors greater 
than half the number of data points render the graph virtually feature-less. Try 
values between 2 and 10. The values need not be integers.

The {\bf Fit/Rexx Fit} option will not be described here. It works in the same way 
as regression fitting, but allows you to specify your own function to be fitted 
using ARexx. See Chapter~\ref{ch:rxfit} for details.








\section{Making Plots}
To produce a file for plotting\index{Plotting}, simply select
the {\bf PostScript}, {\bf HPGL}, or {\bf IFF-DR2D} subitem from the {\bf Plot} 
item of the {\bf Project} menu  or use the keyboard shortcuts, \RA{P}, \RA{H} or 
\RA{I} respectively.
A file requester like that you saw for loading a file will appear. Here you 
specify the filename for your plot file and select the {\bf OK} gadget. 
If you produce 
PostScript output, the file may then be sent to a PostScript printer or processed 
by a PostScript interpreter such as PixelScript, Post, or SaxonScript. HPGL files 
are sent to an HPGL pen plotter or processed by the public domain PLT:\ interpreter.
IFF-DR2D plots must be imported into a suitable desktop publishing program or 
structured drawing program.

If you have a PostScript printer attached to your Amiga, you may simply specify {\tt 
SER:}\ or {\tt PAR:}\ (depending on to which port your printer is connected) as 
the PostScript filename (with no directory); if you have the SaxonScript 
interpreter, {\tt PSC:}\ may be specified. Similarly, with HPGL plots, you may 
specify {\tt SER:}\ or {\tt PAR:}\ or, if using the public domain interpreter, you 
may simply specify the filename as {\tt PLT:}.









\section{Pens, Colours and Lines}
\label{sect:pens}
You\index{Pens} will have noticed that the grid has been displayed 
in different colours from the data line and the axes. \amplot\ allows you to specify 
6 pens. Select the {\bf Set Pens} item from the {\bf Project} menu. In the 
requester which appears, you will see that each pen has a colour and a thickness. 
The graph's axes are always drawn in Pen~6, while the grid is drawn in Pen~5. The 
graph which you have displayed on the screen at present has the data drawn in 
Pen~1. Experiment with changing the colours of Pens~1,5 and~6, using only values 
between 1 and~3. (You may specify large numbers which will be used to select pens for 
HPGL output, but the screen display will automatically be set to a number between 1 
and~3.) The specification of thickness relates to PostScript and IFF-DR2D
output and the value is given in `points' (1pt $= \frac{1}{72}$inch). Select 
the {\bf OK} gadget to redisplay the graph with your chosen colours.

Reset\index{Pens} the graph style to a line graph
({\bf XY} from the {\bf Style} menu) and
switch any smoothing or fitting options off.
The data is currently drawn using Pen~1, this may be changed using the {\bf Pen} 
item of the {\bf Lines} menu. When you select this item, a requester will appear 
in which you may enter a pen number between 1 and~6. Set the pen number to 2, 
select {\bf OK} (or hit return) and click on one of the line's datapoints.
The line will be replotted 
using Pen~2 using the colour set for Pen~2 using the {\bf Project/Set Pens} menu item.
Note that the data point marks have been retained when you switched to the line 
graph and that they reverted to crosses when you select a pen number other 
that 1. By selecting the {\bf Data Points} item from the {\bf Style} menu, you 
may map each pen to a different data point mark.

To\index{Pens} demonstrate the use of Pens in more detail,
we will load a new dataset which has two sets of data.
From the {\bf Project} menu, select the {\bf Open} item and load {\tt demo2.dat}.
You will now see 2 lines plotted on the screen. The first of these is plotted in 
Pen~1, the second in Pen~2. Once again, by using the {\bf Pen} item from the {\bf 
Lines} menu, you may set either line to a different pen number. Experiment with 
this now, ending up with one line in Pen~1 and the other in Pen~2.

At\index{Data Points, Line Setup} this stage, all data points will be switched off.
Select the {\bf Line Setup} item from the {\bf Style} menu. You will see all the 
data points set to blanks. 
Select {\bf Cancel} from this requester and select the {\bf Scatter} item from 
the {\bf Style} menu. You will now see the two graphs plotted as scatter plots 
with crosses. Selecting {\bf Style/Line Setup} once again will show all the Pens have 
been set to crosses rather than blanks. The Scatter option automatically sets any
Pens to crosses which have been left as blanks. Change Pen~1 to open circles and 
Pen~2 to filled circles and select {\bf OK}. Use the {\bf Project/Set Pens} 
requester to set both Pens 1 and~2 to a line thickness of 1pt and colour~1.
Set the X-axis to a log scale and turn on regression smoothing with a polynomial 
degree of 7. Create a title using the {\bf Text/Title} requester.

We shall now create a key for the graph. Select {\bf Key} from the {\bf Text} menu.
Set the key position to X: 7.0, Y: 40.0. Now enter a key label in the string gadget 
labelled {\bf Text} for the first dataset. For this example, we'll call it 
`Device A'. Hit the return key, or click the {\bf Next} gadget; the dataset number 
will change to 2. Enter `Device B' in the {\bf Text} gadget. Now click on the {\bf 
OK} gadget. A key will now appear on the graph. Try producing a plot of your graph.








\section{Robust Fitting Demonstration}
\label{sect:rob}
Robust\index{Robust Fitting} fitting is useful when a set of data
which should fit a straight line 
contains a number of outlying points which disrupt linear regression
\index{linear regression} analysis.
To show the difference between the two fitting methods, load the file {\tt demo3.dat}
into \amplot. A scatter plot will be displayed with two fitted lines. The 
data points clearly fit a straight line although there are a number of outlying 
points. These influence the regression fitting. The robust straight line fit, 
however, is much less influenced by these outliers and fits the major set of 
points much better.








\section{Bar Charts}
\label{sect:bar}
Once again, load\index{Bar Charts} the file {\tt demo2.dat} into \amplot\ and 
select the {\bf Bar} style from the {\bf Style} menu. The two sets of data will 
be displayed as a bar chart. The first set of data is displayed in  Pen~1 with the 
second set in Pen~2. Pens may be changed as before. Note also that the second set 
of data has bars slightly narrower than the first set. Select the {\bf Bar Setup} 
item\index{Bar Style} from the {\bf Style} menu. A requester will appear allowing 
you to control various aspects of the barchart's style. For example you may set the 
factor by which shrinkage\index{Bar shrinkage} of the bars between datasets will 
occur. Setting this to zero will prevent any shrinkage. Conversely selecting 
{\bf Shrink first data set} \index{Bar shrinkage} will cause the first dataset to 
have shrunken bars as well.
Selecting the {\bf Grouped} \index{Bar Grouped} gadget will change the layout of 
the bars (when using this feature, the {\bf Shrink first data set} checkbox has
no effect). Try this now. The barchart will be replotted with the bars from the 
two datasets in pairs.

Select the {\bf Style/Bar Style} menu item again and select {\bf Outline 
chart}\index{Bar Outline}. When you select {\bf OK}, the barchart will be replotted 
as an outline. This is useful if you are trying to represent the area under the bars.
Finally, selecting {\bf Style/Bar Style} again and selecting the {\bf Stacked} 
gadget will create a stacked barchart. This is most effective with the {\bf Shrink 
first data set} checkbox set and with fills used in the bars. Note that the Y-axis 
of the graph will be rescaled to fit the stacked bars.

Select the {\bf Axis \& Pie Labels} item from the {\bf Text} menu.
As well as allowing you to control the font used for the axis labels, this requester 
allows you to switch off the axis labels and to centre the labels on the X-axis. 
This is useful for barcharts. In addition, this requester allows you to place text 
labels along the X-axis rather than numbers. Click on the {\bf Bar/Pie Labels }
gadget at the bottom of the requester. A further requester will appear with a single 
text gadget and {\bf Prev} and {\bf Next} gadgets. You can type a label for each bar 
in the text gadget. Pressing the return key will have the same effect as clicking on 
the {\bf Next} gadget. Enter some text for each of the bars. When you have finished 
click the {\bf OK} gadget. The main {\bf Axis Labels} requester will now have the 
{\bf No X-labels} checkbox set. Click the {\bf OK} gadget. The normal X-axis labels 
will disappear and the text labels will appear instead.

In the overlayed, grouped and stacked bar styles, the bars may be filled using 
the {\bf Fills} item\index{Bar Fills} from the {\bf Lines} menu. 
Choose the {\bf Overlayed} style from the {\bf Style/Bar Setup} requester.
Select one of the fills from the sub-items of the {\bf Lines/Fills} menuitem.
The pointer will change to the click `target'. Click with the 
left mouse button inside one of the bars.
All the bars in this dataset will be filled. Do the same 
with the second dataset, selecting a different fill. You may have noticed that 
the {\bf Fill all bars} checkbox\index{Fill All} in the {\bf Style/Bar Setup} 
requester has been set. If you click this checkbox to remove the checkmark,
only the actual bars in which you clicked will be filled. This is most useful when 
you only have one dataset and wish to fill each bar differently. The current fills 
may be switched off by choosing the {\bf No Fills} item from the {\bf Lines} menu.

Options such as the grid, setting the tick 
marks, bounds, titles, labels, etc.\ are all still available to you.
Try some of these options for yourself.







\section{Pie Charts}
\label{sect:pie}
Load\index{Pie Charts} the file {\tt demo4.dat} into \amplot. The data will be 
displayed 
as a Pie chart. Up until now, the {\bf Pie} item from the {\bf Style} menu has been 
disabled. Your dataset must contain the keyword {\tt PIE} of {\tt Columns 1} to 
enable the piechart option. Data to be presented as a piechart tends to be of a 
different form from data you would display in another form so this has been done to 
prevent meaningless graphs. (Note that it is possible to set the pie style for 
datasets which do not have one of these keywords by using  the ARexx interface. See 
Chapter~\ref{ch:datafile} and Chapter~\ref{ch:rxfit} for details.)

As with bar charts, fills\index{Pie Fills} may be used.
The {\bf Axes/Boxed} menu item will create a box around the chart. The box will 
match the dimensions specified in the {\bf Project/Paper} requester exactly.
To alter the size of the piechart with respect to the box, making room for a key 
and/or a title, the {\bf Axes/Set Axes} requester should be used. The piechart 
itself has a radius of 100\index{pie dimensions} units. The box defaults to minimum 
and maximum values of $\pm120$. These values may be increased to make the 
pie\index{pie boxing} smaller within the box.

The {\bf Text/Extra labels} menu item may be used to add arbitrary labels to the 
piechart. The centre of the pie is at coordinate 0.0,~0.0 and the radius of the 
circle is 100.0 units. The {\bf Text/Axis \& Pie Labels} requester may be used to 
add a label next to each slice of the pie. Select this menu item and click the {\bf 
Bar/Pie labels} gadget. Into the requester which appears, you can enter a label for 
each  slice of the pie. Click {\bf OK} on each of the requesters and the labels will 
appear next to the pie slices.

Slices may be ejected from the pie using the {\bf Eject Slice} item of the {\bf 
Style} menu. Select this item and click on a slice of the pie. The slice will be 
ejected from the pie and may be replaced by repeating the procedure. The {\bf Pie 
Setup} item of the {\bf Style} menu allows you to specify the percentage of the size 
of the piechart by which the slice is ejected. It also allows you to specify the 
angle used for the start of the piechart. By default, this is 0\degree\ which is 
the horizontal right-pointing axis. 90\degree\ specifies the up-pointing vertical 
axis. Any number between 0.0 and 360.0 may be given.

Labels added with the {\bf Text/Axis \& Pie Labels} requester will move with the pie 
when you change the start angle. Labels added with the {\bf Text/Extra labels} 
requester will {\em not\/} move with the pie start angle but are fixed with respect 
to the graph `axes'.

